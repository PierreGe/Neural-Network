\documentclass[a4paper,10pt]{article}
\usepackage[french]{babel}
\usepackage[utf8]{inputenc}
\usepackage[left=2.5cm,top=2cm,right=2.5cm,nohead,nofoot]{geometry}
\usepackage{url}
\usepackage{graphicx}
\usepackage{float}
\usepackage[colorinlistoftodos]{todonotes}
\usepackage{hyperref}
\usepackage{amssymb}
\usepackage{dsfont}
\usepackage{amsmath}

\linespread{1.1}



\begin{document}

\begin{titlepage}
\begin{center}
\textbf{\textsc{UNIVERSIT\'E DE MONTR\'EAL}}\\
%\textbf{\textsc{Faculté des Sciences}}\\
%\textbf{\textsc{Département d'Informatique}}
\vfill{}\vfill{}
\begin{center}{\Huge Rapport : Devoir3 }\end{center}{\Huge \par}
\begin{center}{\large Pierre Gérard \\ Mathieu Bouchard}\end{center}{\Huge \par}
\vfill{}\vfill{} \vfill{}
\begin{center}{\large \textbf{IFT3395-6390 Fondements de l'apprentissage machine}}\hfill{\\Pascal Vincent, Alexandre de Brébisson et César Laurent}\end{center}{\large\par}
\vfill{}\vfill{}\enlargethispage{3cm}
\textbf{Année académique 2015~-~2016}
\end{center}
\end{titlepage}

%\begin{abstract}
%Ce rapport présente ...
%\end{abstract}


%\tableofcontents

%\pagebreak

\section{Partie pratique : Implémentation du réseau de neurones}

\subsection{Exercices 1 et 2}

\subsection{Exercices 3 et 4	}


\end{document}
